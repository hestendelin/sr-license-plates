\section{Введение}
Цифровое изображение -- это спроецированное на матрицу фотоаппарата изображение реального мира. Из-за особенности
цифровой техники это изображение получается с конечным количеством пикселей. Количество этих пикселей деленное на
единицу площади называется разрешением цифрового изображения.

Для анализа и обработки изображений в большинстве случаев используются изображения высокого разрешения, так как они
позволяют увидеть детали, которые не различимы, или плохо различимы на изображениях с низким разрешением. Многие задачи
компьютерного зрения изначально полагаются на то, что изображение на входе в хорошем разрешение. С первого взгляда,
задача увеличение разрешения чисто аппаратная -- чтобы увеличить разрешение необходимо просто взять фотоаппарат с
б\'oльшей разрешающей способностью. Но часто уже имеется некоторая фотография или набор фотографий, и нет возможности
повторно произвести съемку. Примером может послужить низкокачественные картинки с изображений камер наблюдения. В этом
случае уместно использовать алгоритмы <<Super-resolution>>. Эти алгоритмы используют некоторое дополнительное знание о
изображении, для того, чтобы качественно повысить разрешение. Примером дополнительных знаний может послужить информация
о том, как двигался объект во время съемки, или оптические параметры камеры.

В этой работе рассмотрены два алгоритма для повышения разрешения на примере автомобильных номеров.

\section{Обзор существующих алгоритмов}
Классические алгоритмы Super-Resolution делятся на пять категорий.  Существует
несколько обзорных статей по этим алгоритмам \cite{ParkS.C.2003,tian2011survey}

\subsection{Обучаемые алгоритмы}
Обучаемые алгоритмы используют для восстановления изображений используют знание о том,
что в действительности изображено на картинке. Такие алгоритмы хорошо подходят для повышения разрешения однотипных
изображений, таких как лица или автомобильные номера.

\subsection{Интерполяционные}
Подход интерполяции наиболее интуитивно понятный метод решения задачи SR. Решение задачи
разбивается на три шага: определение относительного движения (Motion estimation) для каждого изображения, неравномерная
интерполяция LR изображений на HR сетку, удаление смазывания и шума для полученного HR изображения.

\subsection{Спектральное представление}
В алгоритмах этого класса используется разложение изображения в некоторый другой
базис (примером может являться двумерное дискретное преобразование Фурье, или дискретное косинусное преобразование), и
обратное к нему.  Этот метод удобен тем, что мелкие детали соответствуют высоко частотным компонентам базиса, и за счет
этого становится возможным восстанавливать мелкие детали изображения, которые возможно получить за счет интерполяции с
нескольких LR изображений.

\subsection{Регуляризация} Алгоритмы с использованием регуляризации используют некоторые знания о природе изображения.
Например, в алгоритме MAP используют предположение о вероятности распределений точек.

\section{Постановка задачи}
Для рассуждений необходимо некоторым образом формально записать то, с чем имеем дело. Для

\subsection{Метрика PSNR}
Для того, чтобы сравнивать несколько алгоритмов повышения разрешения необходимо ввести какую-нибудь количественную
метрику. Чтобы иметь возможность посчитать метрику необходимо знать эталонное изображение, то есть такое изображение,
которое должен выдать идеальный алгоритм повышения разрешения. Очевидно, что создать эталонный алгоритм невозможно --
это отображение из $\mathbb{N}^{n \cdot m } \to \mathbb{N}^{nk \cdot mk}, k \ge 2$. Знание эталонного изображения
несколько выходит за рамки поставленной во введении задачи -- в реальных условиях истинное изображение будет неизвестно.
Для тестирования алгоритмов были использована стандартная модель получения изображений низкого разрешения из высокого.

Пусть $x$ -- истинное изображение, $\tilde{x}$ -- найденное изображение из изображений низкого разрешения. Посчитаем
среднеквадратичную ошибку между пикселями этих изображений.

$$ \mathrm{MSE}(\tilde{x},x) = \frac{1}{m\,n}\sum_{i=0}^{m-1}\sum_{j=0}^{n-1} [\tilde{x}(i,j) - x(i,j)]^2$$

И обозначим величину обратную ей и выраженную на логарифмической шкале как $\mathrm{PSNR}(\tilde{x},x)$.
$$ \mathrm{PSNR}(\tilde{x},x) &= 10 \cdot \log_{10} \left( \frac{\mathrm{MAX}_I^2}{\mathrm{MSE}(\tilde{x},x)} \right) $$
Это и будет нашей метрикой. Задачей будет достигнуть максимального значения PSNR из имеющихся данных.

\subsection{Интерполяция}

Одним из наиболее известных способов повышения разрешения является интерполяция. Для любого изображения можно бесконечно
повышать разрешения, просто добавляя дополнительные значения между пикселями исходного изображения. Однако, такое
повышение разрешения не всегда дает хорошие результаты. Поскольку значение PSNR лишь показывает разницу между двумя
изображениями, то нет возможности численно оценить насколько алгоритм Super-resolution хорошо справился с задачей. Но
если сравнивать этот же алгоритм со значением полученные каким-либо методом интерполяции то наглядно видно, где алгоритм
справляется хорошо, а где плохо.

\subsection{Изображения как векторы}

Для возможности записывать все операции над изображением как матричное умножение, будем считать, что изображение
высокого разрешения, которое мы хотим восстановить $x$ размера $L_{1}N_{1}\times L_{2}N_{2}$ записано в виде вектора в
виде вектора $x=[x_{1},x_{2},\dots,x_{N}]^{T}$, где $N=L_{1}{N_{1}\times L_{2}N_{2}}$. Другими словами, изображение $x$
это то изображение, которое мы бы получили, если бы использовали матрицу фотоаппарата с более большим разрешением. Мы
будем рассматривать следующую стандартную модель искажения изображения $$\[ y_{k}=DBM_{k}x+\sigma_{k}\epsilon \]$$ где
$M_{k}$матрица геометрического преобразования для конкретного изображения $y_{k}$, $B_{k}$ обозначает матрицу размытия
размером $L_{1}N_{1}L_{2}N_{n}\times L_{1}N_{1}L_{2}N_{2}$, $D$ матрица понижения разрешения размера,
$\sigma_{k}\epsilon$-- аддитивный шум.

\section{Используемые алгоритмы}

Для того, чтобы повысить разрешение автомобильного номера использовались два подхода: обучаемый \cite{yang2012coupled} и
с использованием регуляризации \cite{suresh2007superresolution}. Этот выбор основывался исходя из природы изображения
автомобильного номера.

\subsection{Обучаемый алгоритм на словарях}
В статье\cite{yang2012coupled}
тип алгоритма -обучаемый

обоснование применения

про словарь

принцип обучения

принцип работы

\subsection{Алгоритм с использованием регуляризации}

тип алгоритма регуляризационный

обоснование применения

MAP/ GIBS

принцип оптимизации

критерий сходимости

подбор параметров

\section{Результаты}

\subsection{Обучаемый алгоритм на словарях}

графики сравнения PSNR скорость работы примеры изображений

\subsection{Алгоритм с использованием регуляризации }

графики сравнения PSNR скорость работы примеры изображений

сравнение с билинейной/бикубической интерполяцией

\section{Заключение}

\subsection{какие средства использовались}

\subsection{что было сделано}

\subsection{Дальнейшее исследование}

\subsubsection{исследование других подходов}

\subsubsection{автоматическое детектирование}

\newpage
\nocite{*}
\bibliographystyle{plainnat}
\bibliography{bib}

