
\section*{Введение}

Для анализа и обработки изображений в большинстве случаев используются
изображения высого разрешения. Разрешения изображения определяет количество
различимых деталей, содержащихся в изображении. Благодаря этому, возможно
рассмотреть детали, которые бы мы не увидели бы в изображении низкого
разрешения.

Есть несколько путей получения изображений высокого разрешения. Изначально,
цифровое изображение получается тем, что проектируя изображения из
реального мира на матрицу фотоаппарата мы получаем цифровое изображение
фиксированного размера. Если увеличить размер матрицы фотоаппарата,
то и разрешения изображения тоже увеличится.

Другой путь, это восстановление изображений высокого качества из изображний
низкого качества.


\section{Модель наблюдений}

Для рассуждений необходимо некоторым образом формально записать то,
с чем имеем дело. Будем считать, что HR изображение, которое мы хотим
восстановить $x$ размера $L_{1}\dot{N_{1}\times L_{2}N_{2}}$ записано
в виде вектора в лексиграфическом порядке в виде вектора $x=[x_{1},x_{2},\dots,x_{N}]^{T}$,
где $N=L_{1}\dot{N_{1}\times L_{2}N_{2}}$. Говоря по другому, изображение
$x$ это то изображение, которое мы бы получили, если бы использовали
матрицу фотоаппарата с более большим разрешением. Чтобы продолжать
рассуждения, необходимо сделать несколько предположений о наблюдаемом
объекте, которое позволит нам восстановить изображение. Будем рассматривать
модель преобразования
\[
y_{k}=DBM_{k}x+\sigma_{k}\epsilon
\]


где $M_{k}$матрица геометрического преобразования для конкретного
изображения $y_{k}$, $B_{k}$ обозначает матрицу размытия размером
$L_{1}N_{1}L_{2}N_{n}\times L_{1}N_{1}L_{2}N_{2}$, $D$ матрица понижения
разрешения размера , $\sigma_{k}\epsilon$-- аддитивный шум.


\section{Обзор существующих методов}

Классические алгоритмы Super-Resolution делятся на пять категорий.
Существует несколько обзорных статей по этим алгоритмам \cite{ParkS.C.2003,TianJ.2011}


\subsection{Обучаемые алгоритмы}

Обучаемые алгоритмы используют для восстановления изображений используют
знание о том, что в действительности изображено на картинке. Такие
алгоритмы хорошо подходят для повышения разрешения однотипных изображений,
таких как лица или автомобильные номера.


\subsection{Интерполяционные}

Подход интерполяции наиболее интуитивно понятный метод решения задачи
SR. Решение задачи разбивается на три шага: определение относительного
движения (Motion estimation) для каждого изображения, неравномерная
интерполяция LR изображений на HR сетку, удаление смазывания и шума
для полученного HR изображения.


\subsection{Спектральное представление}

В алгоритмах этого класса используется разложение изображения в некоторый
другой базис (примером может являться двумерное дискретное преобразование
Фурье, или дискретное косинусное преобразование), и обратное к нему.
Этот метод удобен тем, что мелкие детали соответсвуют высоко частотным
компонентам базиса, и за счет этого становится возможным восстанавливать
мелкие детали изображения, которые возможно получить за счет интерполяции
с нескольких LR изображений.


\subsection{Регуляризация}

Алгоритмы с использованием регуляризации используют некоторые знания
о природе изображения. Например, в алгоритме MAP используют предположение
о вероятности распределений точек.


\section{Текущее состояние}

Запускаю BM3D фильтр на нескольких изображениях и меряем PNSR. Что-то
пытался изменять, но не очень успешно. Пытаюсь осознать BM3D с математической
стороны.

\newpage{}

\begin{comment}\bibliographystyle{gost780s}
\phantomsection\addcontentsline{toc}{section}{\refname}\nocite{*}
\bibliography{bib}


\end{comment}
