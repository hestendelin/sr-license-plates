% \title{ Сравнение подходов обучения на базе словаря и MAP к проблеме повышения разрешения на примере изображений автомобильных номеров }
\title{Повышение разрешения изображений автомобильных номеров}
\author{
  \begin{tabular}[4cm]{rl}
 Автор:                & Улитин А.~А., 461 гр. \\
 Научный руководитель: & доц.~Вахитов А.~Т. \\
 Рецензент: & Пименов А. А. \\
 \end{tabular}
 }
\date{2013 год}

\begin{frame}{}
		\maketitle
\end{frame}

\section{Задача}
\begin{frame}{Задача Super-resolution}
  Задача Super resolution --- качественно повысить разрешения изображения.
  \includegraphics[height=\textheight]{content/An_example_of_super_resolution_with_still_RAW_photo.jpg}
\end{frame}

\begin{frame}{Почему это возможно}
  Для повышения разрешения используется дополнительная информация
  \begin{itemize}
    \item Знание параметров съемки (размытие, движение камеры~и~т.п.)
    \item Знание о типе снимаемого объекта (текст, лица, и т.п.)
    \item Использование нескольких изображей, снятых с разных ракурсов
  \end{itemize}
  которая влияет на конечное изображение

  Применимость:
  \begin{itemize}
    \item Препроцессинг для других алгоритмов компьютерного зрения
    \item Извлечение дополнительной информации с нескольких снимков для получения одного кадра с высоким разрешением
  \end{itemize}
\end{frame}

\begin{frame}{PSNR}

  $$ \mathrm{MSE}(\tilde{x},x) = \frac{1}{m\,n}\sum_{i=0}^{m-1}\sum_{j=0}^{n-1} [\tilde{x}(i,j) - x(i,j)]^2$$

  И обозначим величину обратную ей и выраженную на логарифмической шкале как $\mathrm{PSNR}(\tilde{x},x)$.
  $$ \mathrm{PSNR}(\tilde{x},x) &= 10 \cdot \log_{10} \left( \frac{\mathrm{MAX}_I^2}{\mathrm{MSE}(\tilde{x},x)} \right)
  $$

\end{frame}

\begin{frame}{Постановка задачи}
 $$y_r = D H_R W_R x + n_r,~ ~ ~ 1 \leq r \leq m$$
 где:
 \begin{itemize}
   \item $x$ оригинальное изображение
   \item $y_r$ наблюдение r
   \item $D$ матрица понижение разрешения
   \item $W$ матрица геометрического искажения
   \item $H_R$ матрица размытия наблюдения r
   \item $n_r$ шум наблюдения r
   \item $m$ количество наблюдений
 \end{itemize}
 Задача найти
 $$ \tilde{x} = \underset{\hat{x}}{\operatorname{argmax}}~  PSNR(\hat{x},x)$$
\end{frame}
\section{Подходы}
\begin{frame}{Сравниваемые подходы}
  \begin{itemize}
    \item Couple Dictionary Training for Image Super-resolution \\
        (Jianchao Yang,Zhaowen Wang, Zhe Lin,Scott Cohen, and Thomas Huang) \\
  Основная идея метода очень проста -- тренировка словаря из патчей небольшого размера в двух разрешениях LR и HR
      \begin{itemize}
        \item Использует пару тренированных словарей
        \item Восстановление по одному изображению
      \end{itemize}
    \item Superresolution of License Plates in Real Traffic Videos \\
      (K. V. Suresh, G. Mahesh Kumar, and A. N. Rajagopalan) \\
      Основная идея -- использование шаговой оптимизации с адаптивным регуляризатором.
      \begin{itemize}
        \item Для восстановление использует последовательную оптимизацию с
          регуляризаторами
        \item Использует несколько изображений
      \end{itemize}
  \end{itemize}
\end{frame}

%\section{Описание подходов}
%\begin{frame}{Подход с тренированными словарями}
%\end{frame}

% \newcommand{\inimage}[2]{
%   \begin{minipage}{#1}
%     \vcenter{\includegraphics[width=\columnwidth]{#2}}
%   \end{minipage}
% }
%
%
% \section{Приведение изображений}
% \begin{frame}{Несколько изображение $\rightarrow$ одно изображение}
%   Для тестирования на однинаковых наборах данных из lr строились псевдо hr для
%   первого алгоритма методом:
%   $$ R = \frac{1}{n}\sum_{i=1}^nW^T \cdot S \cdot IMG_{lr i}$$
%   $S$ --- билинейная интерполяция \\
%   $W$ --- сдвиг
%
%   \inimage{3cm}{content/imgs/append_imgs_big.jpg}
%   $\to$
%   \inimage{6cm}{content/imgs/combined_big.jpg}
% \end{frame}

\begin{frame}{Исходные изображения}
  \includegraphics[width=\columnwidth]{content/out_sr1.png}
\end{frame}

\section{SR1}
\begin{frame}{Результаты подхода алгоритма со словарями}
  \input{../plots/plot_sr1.tex}
\end{frame}

\begin{frame}{Пример изображений алгоритма со словарями}
  \includegraphics[width=\columnwidth]{content/compare_result_sr1.pdf}
\end{frame}

\section{SR2}
\begin{frame}{Результаты подхода}
  % This file was created by matlab2tikz v0.3.3.
% Copyright (c) 2008--2013, Nico Schlömer <nico.schloemer@gmail.com>
% All rights reserved.
% 
% The latest updates can be retrieved from
%   http://www.mathworks.com/matlabcentral/fileexchange/22022-matlab2tikz
% where you can also make suggestions and rate matlab2tikz.
% 
% 
% 
\begin{tikzpicture}

\begin{axis}[%
width=10cm,
height=7.88709677419355cm,
scale only axis,
xmin=0,
xmax=16,
xlabel={Номер изображения},
ymin=12,
ymax=35,
ylabel={PSNR dB},
legend style={draw=black,fill=white,legend cell align=left}
]
\addplot [
color=blue,
solid
]
table[row sep=crcr]{
1 27.4470571608433\\
2 22.8300247931548\\
3 22.1109193658809\\
4 22.0652635937327\\
5 23.6272496813018\\
6 21.6055079117517\\
7 21.4619565703478\\
8 26.9867634881452\\
9 20.9809477047241\\
10 23.2957078121211\\
11 23.9394056382537\\
12 25.5317267141373\\
13 29.2416356490523\\
14 24.1599986234539\\
15 23.7307351145084\\
16 22.5888532577419\\
};
\addlegendentry{Алгоритм с регуляризацией};

\addplot [
color=green!50!black,
solid
]
table[row sep=crcr]{
1 18.4503335459593\\
2 15.9429403769346\\
3 15.9459841548511\\
4 15.5934814518383\\
5 16.9838249182421\\
6 13.7171337206461\\
7 12.7696291928783\\
8 21.6627180369299\\
9 13.6838913514568\\
10 18.8295593160528\\
11 17.4771261841067\\
12 20.0919833166445\\
13 22.2356695114555\\
14 16.697098012576\\
15 18.4542205359683\\
16 15.4386868770896\\
};
\addlegendentry{Начальное приближение};

\end{axis}
\end{tikzpicture}%

\end{frame}

\begin{frame}{Пример изображений алгоритма с регуляризацией}
  \input{../plots/sr2_psnr_rising.tex}
\end{frame}

\begin{frame}{Пример изображений алгоритма с регуляризацией}
  \input{../plots/sr2_psnr_rising_2.tex}
\end{frame}

\begin{frame}{Пример изображений алгоритма с регуляризацией}
%   \begin{tikzpicture}
%     \node (img1) {\includegraphics[width=10cm]{content/sr2_two_1.png}};
%     \node (img1) {\includegraphics[width=10cm]{content/sr2_two_2.png}};
%   \end{tikzpicture}
  \begin{figure}
    \caption{Результат}
    \includegraphics[width=\columnwidth]{content/sr2_two_1.png}\\
  \end{figure}
  \begin{figure}
    \caption{Начальное приближение}
    \includegraphics[width=\columnwidth]{content/sr2_two_2.png}
  \end{figure}
 %\includegraphics[width=\columnwidth]{content/sr2_two_images.pdf}
\end{frame}

\section{Результаты}

\begin{frame}{Результаты}
  \begin{itemize}
    \item Были исследованы и реализованы два подхода к задаче SR
    \item Эксперименты показывают, что оба алгоритма качественно повышают разрешения автомобильного
      номера.
    \item Результаты работы будут опубликованы на конференции СПИСОК-2013
  \end{itemize}
\end{frame}
